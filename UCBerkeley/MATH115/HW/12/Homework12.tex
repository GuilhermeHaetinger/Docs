% Created 2019-12-09 Mon 11:01
% Intended LaTeX compiler: pdflatex
\documentclass[11pt]{article}
\usepackage[utf8]{inputenc}
\usepackage[T1]{fontenc}
\usepackage{graphicx}
\usepackage{grffile}
\usepackage{longtable}
\usepackage{wrapfig}
\usepackage{rotating}
\usepackage[normalem]{ulem}
\usepackage{amsmath}
\usepackage{textcomp}
\usepackage{amssymb}
\usepackage{capt-of}
\usepackage{hyperref}
\usepackage{minted}
\usepackage[margin=0.5in]{geometry}
\author{MATH 115 \\ UC Berkeley \\ Guilherme Gomes Haetinger}
\date{\today}
\title{\huge Homework 12}
\hypersetup{
 pdfauthor={MATH 115 \\ UC Berkeley \\ Guilherme Gomes Haetinger},
 pdftitle={\huge Homework 12},
 pdfkeywords={},
 pdfsubject={},
 pdfcreator={Emacs 27.0.50 (Org mode 9.2.6)}, 
 pdflang={English}}
\begin{document}

\maketitle
\begin{center}
\emph{Thank you for the semester long grading, Zehao! Happy holidays!}
\end{center}

\begin{center}
\line(1,0){250}
\end{center}

\section*{9.1}
\label{sec:org38b6fcc}
\subsection*{5)}
\label{sec:org4f3ca33}
We can state, for \(r \in \mathbb{Q}\), that \(f = hg = r*h_1g^*_1, g,h \in \mathbb{Z}[x]\). By Thm 9.6, we know that \(f_1 = h_1g^*_1\) is primitive. Now  we want to prove that \(r\) \(\in\) Z\$. We first write that \(r = \frac{a}{b}, (a, b) = 1\) and try to prove \(b=1\). By this, we know \(bf = af_1\) and, thus, that \(b|af_1 \to b|f_1\) (\emph{b} divides all the coefficients in \(f_1\)) and, since \(f_1\) is primitive, \emph{b} must equal \emph{1}. Now we can just simply write \(g_1 = r * g^*_1, f = g_1h_1\).
\subsection*{6)}
\label{sec:org69551fa}
We know that \emph{f} and \emph{g} are primitive, so \(f,g \in  \mathbb{Z}[x]\). Following this, we have \(f(x)|g(x), g(x)|f(x) \to g(x) = q(x)f(x), f(x) = q^*(x)g(x) \to f(x) = q^*(x)q(x)f(x) \to q^*(x)q(x) = \pm 1\) (because they must be integers) \(f(x) = \pm g(x)\).
\subsection*{7)}
\label{sec:org84e224d}
Within our claim, we know, by Thm. 9.1, that \(g(m) = q(m)f(m) + r(m)\). We can multiply both sides by an integer \emph{k}, making \(kq(m), kr(m) \in \mathbb{Z}\). Considering that \(g(m) > kr(m)\) for a big enough \(m\), we can state that, since \(g(m) = kq(m)f(m) + kr(m)\) holds all conditions, then so does \(g(m) = q(m)f(m) + r(m)\).
\subsection*{8)}
\label{sec:orgf20d320}
They can be the following: \(f(x) = 2x + 4 = 2(x+2), g(x) = 3x + 3 = 3(x + 1)\). They will always have a GCD of 2 when \emph{x} is odd, meaning they have a \(GCD > 1\) for infinitely many positive integers but their polynomials are coprime.
\subsection*{9)}
\label{sec:orgd2b75aa}
If we take \(x_0 = P^n*f(0)\), as the hint says, with \emph{P} being the product of all finite primes that divide \emph{f}, we have that, for a large \emph{n}, the value \(f(x_0) = P^nf(0)*q(P^nf(0)) + f(0)\) being the constant value larger than 0 will have the following condition: \(f(x_0) > |f(0)|\). This way, we can state \(f(x_0) = f(0)(P^nq(P^nf(0))) + 1)\), meaning that \emph{f} is a polynomial divisible by a number larger than the supposed prime.

\begin{center}
\line(1,0){250}
\end{center}

\section*{9.2}
\label{sec:org02cf9e3}
\subsection*{1)}
\label{sec:org2297dd7}
\begin{itemize}
\item 7 \(\to f(x) = x - 7\)
\item \(\sqrt[3]7 \to \frac{x^3}{7}- 1\)
\item \(\frac{1 + \sqrt[3]7}{2} \to 4x^3 - 6x^2 + 3x - 4\)
\item \(1 + \sqrt2 + \sqrt3 \to x^4 - 4x^3 - 4x^2 + 16x - 6\)
\end{itemize}
We know that only \((\frac{1 + \sqrt[3]7}{2})\) is not an Algebraic Integer.
\subsection*{2)}
\label{sec:orgd566c58}
For the first option \(-\alpha\), we can just use \(alpha\)'s polynomial and switch its sign, meaning the degree would be maintained. The second option, \(\alpha^{-1}\), we can still remodel the original polynomial so it takes in consideration the divisor's magnitude, trivially. The third case \(\alpha - 1\), we can also remodel the coefficients so it takes into consideration the remainder resulted by the powered polynomial \(\alpha - 1\), just canceling each other in every level. 

\begin{center}
\line(1,0){250}
\end{center}

\section*{9.3}
\label{sec:orgbc98d89}
\subsection*{1) Couldn't understand it properly}
\label{sec:orge54390d}
\subsection*{2)}
\label{sec:orgae42a9c}
We have that the minimal polynomial for the items in \(\mathbb{Q}(i)\) is equal \(x^2 + 1\), which is the same as the modulo expression in the other field. Now, because, \(G(x) = x^2 + 1\) is an irreducible polynomial and the statement in the first sentence is true, we have that they are isomorphic corresponding to Thm 9.16.

\begin{center}
\line(1,0){250}
\end{center}

\section*{9.4}
\label{sec:orgcd7ac03}
\subsection*{1)}
\label{sec:orge55e917}
Considering Def. 9.7, we know that the unit of the \(\mathbb{Q}\) field it \(\pm 1\) because \(1 = x\alpha\) with \(x, \alpha \in \mathbb{Z}\) only if \(x, alpha = \pm 1\). Following this definition, we have that for \(\alpha, \beta\) to be associated need to have their values so that \(\frac{\alpha}{\beta} = \pm 1\) (unit), which can only be true if \(\alpha = \pm \beta\).
\subsection*{2)}
\label{sec:orga39400e}
We have that \(m\) is the smallest positive rational so that \(m\alpha\) is an algebraic integer. We also know that every value in the algebraic integer field can be expressed as a multiplication. Since that's true, we know that \emph{b} can be expressed as a multiplication of another field member that is considered the smallest, this way \(b = x * m \to m|b\). 
\subsection*{3)}
\label{sec:orge5453ee}
\begin{itemize}
\item Yes.
\item Not necessarily \(\to \alpha = -\frac{1}{2} + \frac{i\sqrt[3]3}{2} = e^{\frac{2\pi i}{3}}\). \(\alpha\) then satisfies \(x^2 + x + 1\), meaning \(\alpha\) is an algebraic integer even though \(\frac{1}{2}\) isn't.
\end{itemize}
\end{document}
